\documentclass[12pt]{article}
%% arXiv paper template by Flip Tanedo
%% last updated: Dec 2016



%%%%%%%%%%%%%%%%%%%%%%%%%%%%%
%%%  THE USUAL PACKAGES  %%%%
%%%%%%%%%%%%%%%%%%%%%%%%%%%%%

\usepackage{amsmath}
\usepackage{amssymb}
\usepackage{amsfonts}
\usepackage{graphicx}
 

%%%%%%%%%%%%%%%%%%%%%%%%%%%%%%%%%
%%%  UNUSUAL PACKAGES        %%%%
%%%  Uncomment as necessary. %%%%
%%%%%%%%%%%%%%%%%%%%%%%%%%%%%%%%%

%% MATH AND PHYSICS SYMBOLS
%% ------------------------
%\usepackage{slashed}       % \slashed{k}
%\usepackage{mathrsfs}      % Weinberg-esque letters
%\usepackage{youngtab}	    % Young Tableaux
%\usepackage{pifont}        % check marks
%\usepackage{bbm}           % \mathbbm{1} incomp. w/ XeLaTeX 
%\usepackage[normalem]{ulem} % for \sout


%% CONTENT FORMAT AND DESIGN (below for general formatting)
%% --------------------------------------------------------
\usepackage{lipsum}        % block of text (formatting test)
%\usepackage{color}         % \color{...}, colored text
%\usepackage{framed}        % boxed remarks
%\usepackage{subcaption}    % subfigures; subfig depreciated
%\usepackage{paralist}      % compactitem
%\usepackage{appendix}      % subappendices
%\usepackage{cite}          % group cites (conflict: collref)
%\usepackage{tocloft}       % Table of Contents	

%% TABLES IN LaTeX
%% ---------------
%\usepackage{booktabs}      % professional tables
%\usepackage{nicefrac}      % fractions in tables,
%\usepackage{multirow}      % multirow elements in a table
%\usepackage{arydshln} 	    % dashed lines in arrays

%% Other Packages and Notes
%% ------------------------
%\usepackage[font=small]{caption} % caption font is small


%%%%%%%%%%%%%%%%%%%%%%%%%%%%%%%%%%%%%%%%%%%
%%%  FLIP'S CUSTOM PACKAGES            %%%%
%%%  These are in separate .sty files  %%%%
%%%%%%%%%%%%%%%%%%%%%%%%%%%%%%%%%%%%%%%%%%%

%\usepackage{flip-acronyms} % HEP acronyms in small caps, e.g. \GeV
\usepackage{tikzfeynman}   % Flip's rules Feynman Diagrams


%%%%%%%%%%%%%%%%%%%%%%%%%%%%%%%%%%%%%%%%%%%%%%%
%%%  PAGE FORMATTING and (RE)NEW COMMANDS  %%%%
%%%%%%%%%%%%%%%%%%%%%%%%%%%%%%%%%%%%%%%%%%%%%%%

\usepackage[margin=2cm]{geometry}   % reasonable margins

\graphicspath{{figures/}}	        % set directory for figures

\numberwithin{equation}{section}    % set equation numbering
\renewcommand{\tilde}{\widetilde}   % tilde over characters
\renewcommand{\vec}[1]{\mathbf{#1}} % vectors are boldface

\newcommand{\dbar}{d\mkern-6mu\mathchar'26}    % for d/2pi
\newcommand{\ket}[1]{\left|#1\right\rangle}    % <#1|
\newcommand{\bra}[1]{\left\langle#1\right|}    % |#1>
\newcommand{\Xmark}{\text{\sffamily X}}        % cross out

% Change list spacing (instead of package paralist)
% from: http://en.wikibooks.org/wiki/LaTeX/List_Structures#Line_spacing
\let\oldenumerate\enumerate
\renewcommand{\enumerate}{
  \oldenumerate
  \setlength{\itemsep}{1pt}
  \setlength{\parskip}{0pt}
  \setlength{\parsep}{0pt}
}

\let\olditemize\itemize
\renewcommand{\itemize}{
  \olditemize
  \setlength{\itemsep}{1pt}
  \setlength{\parskip}{0pt}
  \setlength{\parsep}{0pt}
}


% Commands for temporary comments
\newcommand{\comment}[2]{\textcolor{red}{[\textbf{#1} #2]}}
\newcommand{\flip}[1]{{\color{red} [\textbf{Flip}: {#1}]}}
\newcommand{\email}[1]{\href{mailto:#1}{#1}}

\newenvironment{institutions}[1][2em]{\begin{list}{}{\setlength\leftmargin{#1}\setlength\rightmargin{#1}}\item[]}{\end{list}}


\usepackage{fancyhdr}		% to put preprint number



% Commands for listings package
%\usepackage{listings}      % \begin{lstlisting}, for code
%
% \lstset{basicstyle=\ttfamily\footnotesize,breaklines=true}
%    sets style to small true-type


%%%%%%%%%%%%%%%%%%%%%%%%%%%%%%%%%%%%%%%%%%%%%%
%%%  TIKZ COMMANDS FOR EXTERNAL DIAGRAMS  %%%%
%%%  requires -shell-escape               %%%%
%%%  in texpad 1.7: prefs > shell esc sec %%%%
%%%%%%%%%%%%%%%%%%%%%%%%%%%%%%%%%%%%%%%%%%%%%%

%% This is for exporting tikz figures as into a ./tikz/ subfolder.
%% It is useful if you want pdf versions of the tikz diagrams or
%% if you need to speed up compilation of a large document with
%% many tikz diagrams.

%\write18{} % Careful with this!
%\usetikzlibrary{external}
%\tikzexternalize[prefix=tikz/] % folder for external pdfs


%%%%%%%%%%%%%%%%%%%
%%%  HYPERREF  %%%%
%%%%%%%%%%%%%%%%%%%

%% This package has to be at the end; can lead to conflicts

\usepackage[
	colorlinks=true,
	citecolor=black,
	linkcolor=black,
	urlcolor=blue,
	hypertexnames=false]{hyperref}



%%%%%%%%%%%%%%%%%%%%%
%%%  TITLE DATA  %%%%
%%%%%%%%%%%%%%%%%%%%%

%% PREPRINT NUMBER USING fancyhdr
%% Don't forget to set \thispagestyle{firststyle}
%% ----------------------------------------------
\renewcommand{\headrulewidth}{0pt} % no separator
\fancypagestyle{firststyle}{
\rhead{\footnotesize \texttt{UCI-TR-2016-XX}}}



\begin{document}

%\thispagestyle{empty}
\thispagestyle{firststyle} %% to include preprint

\begin{center}

    {\huge \bf  Flip's Paper Title}

    \vskip .7cm

    { \bf First Author$^{a}$ and Philip Tanedo$^{b}$ } 
    \\ \vspace{-.2em}
    { \tt
    \footnotesize
    \email{first.author@university.edu}, 
    \email{flip.tanedo@uci.edu} 
    }
	
    \vspace{-.2cm}

    \begin{institutions}[2.25cm]
    \footnotesize
    $^{a}$ {\it 
        Department of Physics, 
        University, 
        University Town, \textsc{ca} 90210
        }
	\\ 
	\vspace*{0.05cm}
	$^{b}$ {\it 
	    Department of Physics \& Astronomy, 
	    University of California, 
	    Irvine, \textsc{ca} 92697
	    }   
    \end{institutions}

\end{center}



%%%%%%%%%%%%%%%%%%%%%
%%%  ABSTRACT    %%%%
%%%%%%%%%%%%%%%%%%%%%

\begin{abstract}
\noindent 
This is a simple template for my papers. It's not very different from the plain article style, but it has most of the macros I use pre-written.
\end{abstract}




%%%%%%%%%%%%%%%%%%%%%
%%%  THE MEAT    %%%%
%%%%%%%%%%%%%%%%%%%%%

% Use \input if you have separate files.
% \include is `smarter' (creates separate aux files for each tex file) 
% and hence more efficient, but it automatically puts a page break
% between included files. Input doesn't do this.


\section{Introduction}

In the beginning...


\section{Note on External TikZ Diagrams}

 This is a bit cumbersome, so I'll leave it commented out by default
 Usage: create the subdirectory tikz (or whatever ``prefix'' you use in
 the \texttt{tikzexternalize} option). Make sure the \texttt{-shell-escape} 
 is used when you compile. 
 %
 For Texpad this is a checkbox in the typesetting preferences.
 Tikz pictures will be exported as PDFs in the tikz directory.

 Note: this is NOT perfect! Some diagrams come out wonky, especially if you
 use arrows (e.g. for Feynman diagrams) or if you put the Tikz pictures in
 odd places (like in equation environments). What it does buy you is a HUGE
 improvement in compile time. I suggest using this for intermediate 
 typesets in a large document. For the final compile just turn it off again
 so you get clean graphics.
 
\section{Itemize test}

\begin{itemize}% \itemsep1pt \parskip0pt \parsep0pt
\item Item 1
\item Item 2
\item Item 3
\end{itemize}

\begin{enumerate}% \itemsep1pt \parskip0pt \parsep0pt
\item Item 1
\item Item 2
\item Item 3
\end{enumerate}


 
\section{lipsum}
 
 \lipsum[3-5]


\section{XeLaTeX}

XeLaTeX lets you access local system fonts for use in \LaTeX. It's great for Beamer, but I don't recommend it in a regular paper. It doesn't play well with some useful macros like `blackboard math,' \url{http://tinyurl.com/a28hrle}.




\section*{Acknowledgements}


This work is supported in part by the \textsc{nsf} grant \textsc{phy}-1316792. 
%
\textsc{p.t.}\ thanks 
\emph{your name here}
for useful comments and discussions. 
%
\textsc{p.t.} thanks the Aspen Center for Physics (NSF grant \#1066293) for its hospitality during a period where part of this work was completed.

%% Appendices
% \appendix

% \bibliographystyle{utphys} 
% \bibliography{bib title without .bib}

\end{document}