%% LaTeX Paper Template, Flip Tanedo (flip.tanedo@ucr.edu)
%% last updated: September 2018

\documentclass[12pt]{article}

%%%%%%%%%%%%%%%%%%%%%%%%%%
%%%  COMMON PACKAGES  %%%%
%%%%%%%%%%%%%%%%%%%%%%%%%%

\usepackage{amsmath}
\usepackage{amssymb}
\usepackage{amsfonts}
\usepackage{graphicx}
\usepackage[utf8]{inputenc}	
%\usepackage{amsthm}				

%%%%%%%%%%%%%%%%%%%%%%%%%%%%%%%%%
%%%  UNUSUAL PACKAGES        %%%%
%%%  Uncomment as necessary. %%%%
%%%%%%%%%%%%%%%%%%%%%%%%%%%%%%%%%

%% MATH AND PHYSICS SYMBOLS
%% ------------------------
%\usepackage{slashed}       % \slashed{k}
%\usepackage{mathrsfs}      % Weinberg-esque
%\usepackage{youngtab}	    % Young Tableaux
%\usepackage{pifont}        % check marks
%\usepackage{bbm}           % chalkboard bold
%\usepackage[normalem]{ulem} % for \sout
%\usepackage{cancel}

%% CONTENT FORMAT AND DESIGN
%% -------------------------
\usepackage[dvipsnames]{xcolor}
\usepackage{fancyhdr}		% preprint number
\usepackage{lipsum}         % block of text
\usepackage{framed}			% boxed remarks
%\usepackage{subcaption}    % subfig depreciated
%\usepackage{paralist}      % compactitem
%\usepackage{appendix}      % subappendices
%\usepackage{cite}          % group cites 
%\usepackage{tocloft}       % Table of Contents	
%\usepackage{xspace}		% macro spacing

%\usepackage{listings}      % for code, eg.
%
% \begin{lstlisting} 
%	\lstset{
%		basicstyle=\ttfamily\footnotesize,
%		breaklines=true,
%		backgroundcolor=\color{gray!15!white}}


%% TABLES IN LaTeX
%% ---------------
\usepackage{booktabs}		% professional
\usepackage{nicefrac}		% fractions
%\usepackage{multirow}     
%\usepackage{arydshln}		% dashed lines


%% Other Packages and Notes
%% ------------------------
\usepackage[font=small]{caption} 
\usepackage{float}         % strict [H]


%% CUSTOM PACKAGES
%% ---------------
%\usepackage{tikzfeynman}   % Flip's rules Feynman Diagrams



%%%%%%%%%%%%%%%%%%%%%%%%%%%%%%
%%%  DOCUMENT PROPERTIES  %%%%
%%%%%%%%%%%%%%%%%%%%%%%%%%%%%%

\usepackage[margin=2cm]{geometry}   
\graphicspath{{figures/}}			
\numberwithin{equation}{section}    

%% References in two columns, smaller
\usepackage{multicol}
\usepackage{etoolbox}
\usepackage{relsize}
\patchcmd{\thebibliography}
  {\list}
  {\begin{multicols}{2}\smaller\list}
  {}
  {}
\appto{\endthebibliography}{\end{multicols}}
%% Alternative (one column, modify spacing):
%% https://wiki.math.cmu.edu/iki/wiki/tips/20140712-bibtex-spacing.html

% Change list spacing
% from: http://en.wikibooks.org/wiki/LaTeX/List_Structures#Line_spacing
\let\oldenumerate\enumerate
\renewcommand{\enumerate}{
  \oldenumerate
  \setlength{\itemsep}{1pt}
  \setlength{\parskip}{0pt}
  \setlength{\parsep}{0pt}
}

\let\olditemize\itemize
\renewcommand{\itemize}{
  \olditemize
  \setlength{\itemsep}{1pt}
  \setlength{\parskip}{0pt}
  \setlength{\parsep}{0pt}
}


%%%%%%%%%%%%%%%%%%%%%%%%%%%
%%%  (RE)NEW COMMANDS  %%%%
%%%%%%%%%%%%%%%%%%%%%%%%%%%

%% FOR `NOT SHOUTING' CAPS (e.g. acronyms)
%% ---------------------------------------
\newcommand{\acro}[1]{\textsc{\MakeLowercase{#1}}}    

%% COMMON PHYSICS MACROS
%% ---------------------
\renewcommand{\tilde}{\widetilde}   % tilde over characters
\renewcommand{\vec}[1]{\mathbf{#1}} % vectors are boldface
\newcommand{\dbar}{d\mkern-6mu\mathchar'26}    % for d/2pi
\newcommand{\ket}[1]{\left|#1\right\rangle}    % <#1|
\newcommand{\bra}[1]{\left\langle#1\right|}    % |#1>
\newcommand{\Xmark}{\text{\sffamily X}}        % cross out

%% COMMANDS FOR TEMPORARY COMMENTS
%% -------------------------------
\newcommand{\comment}[2]{\textcolor{red}{[\textbf{#1} #2]}}
\newcommand{\flip}[1]{{
	\color{green!50!black} \footnotesize [\textbf{\textsf{Flip}}: \textsf{#1}]
	}}


%% COMMANDS FOR TOP-MATTER
%% -----------------------
\newcommand{\email}[1]{\href{mailto:#1}{#1}}
\newenvironment{institutions}[1][2em]{\begin{list}{}{\setlength\leftmargin{#1}\setlength\rightmargin{#1}}\item[]}{\end{list}}


%% COMMANDS FOR LATEXDIFF
%% ----------------------
%% see http://bit.ly/1M74uwc
\providecommand{\DIFadd}[1]{{\protect\color{blue}#1}} %DIF PREAMBLE
\providecommand{\DIFdel}[1]{{\protect\color{red}\protect\scriptsize{#1}}}

%% REMARK: use latexdiff option --allow-spaces
%% for \frac, ref: http://bit.ly/1iFlujR


%%%%%%%%%%%%%%%%%%%%%%%%%%%%%%%%%%%%%%%%%%%%%%
%%%  TIKZ COMMANDS FOR EXTERNAL DIAGRAMS  %%%%
%%%  requires -shell-escape               %%%%
%%%  in texpad 1.7: prefs > shell esc sec %%%%
%%%%%%%%%%%%%%%%%%%%%%%%%%%%%%%%%%%%%%%%%%%%%%

%% For exporting tikz figures as into a ./tikz/ subfolder.
%% It is useful if you want pdf versions of the tikz diagrams or
%% if you need to speed up compilation of a large document with
%% many tikz diagrams.

%\write18{} % Careful with this!
%\usetikzlibrary{external}
%\tikzexternalize[prefix=tikz/] % folder for external pdfs


%%%%%%%%%%%%%%%%%%%
%%%  HYPERREF  %%%%
%%%%%%%%%%%%%%%%%%%

%% This package has to be at the end; can lead to conflicts

\usepackage[
	colorlinks=true,
	citecolor=green!50!black,
	linkcolor=NavyBlue!75!black,
	urlcolor=green!50!black,
	hypertexnames=false]{hyperref}


%%%%%%%%%%%%%%%%%%%%%
%%%  TITLE DATA  %%%%
%%%%%%%%%%%%%%%%%%%%%

%% PREPRINT NUMBER USING fancyhdr
%% Must set \thispagestyle{firststyle}
%% ----------------------------------------------
\renewcommand{\headrulewidth}{0pt} 	% no separator
\setlength{\headheight}{15pt} 		% min to avoid fancyhdr warning
\fancypagestyle{firststyle}{
	\rhead{\footnotesize%
	\texttt{UCR-TR-2019-FLIP-00X}%\\
%	\texttt{UCI-TR-2019-XX}
	}}

%% TOC overwrites fancyhdr, here's a fix
%% http://tex.stackexchange.com/questions/167828/difficult-with-fancyhdr-and-table-of-contents
\usepackage{etoc}
\renewcommand{\etocaftertitlehook}{\pagestyle{plain}}
\renewcommand{\etocaftertochook}{\thispagestyle{firststyle}}



\begin{document}

%\thispagestyle{empty}		% default if no preprint #
%\thispagestyle{firststyle} % to include preprint
							% overwritten by etoc

\begin{center}

    {\huge \bf Clever Paper Title}

    \vskip .7cm

%% SINGLE AUTHOR FORMAT
%% --------------------
	\textbf{Flip Tanedo} \\
	\texttt{\footnotesize \email{flip.tanedo@ucr.edu}}

	\vspace{-1em}
    \begin{institutions}[2.25cm]
    \footnotesize
    {\it 
	    Department of Physics \& Astronomy, 
	    University of  California, Riverside, 
	    {CA} 92521	    
	    }    
    \end{institutions}


%% MULTIPLE AUTHOR FORMAT
%% --------------------
%    { \bf 
%    	Philip Tanedo$^{a}$
%    	and 
%    	Another Author$^{b}$ 
%    	} 
%    \\ 
%    \vspace{-.2em}
%    { \tt \footnotesize
%	    \email{flip.tanedo@ucr.edu},
%	    \email{another.author@university.edu}
%    }
%	
%    \vspace{-.2cm}
%
%    \begin{institutions}[2.25cm]
%    \footnotesize
%    $^{a}$ 
%    {\it 
%	    Department of Physics \& Astronomy, 
%	    University of  California, Riverside, 
%	    {CA} 92521	    
%	    }    
%	\\ 
%	\vspace*{0.05cm}   
%	$^{b}$ 
%	{\it 
%        Department of Physics, 
%        University, 
%        University Town, CA 90210
%        }
%    \end{institutions}

\end{center}




%%%%%%%%%%%%%%%%%%%%%
%%%  ABSTRACT    %%%%
%%%%%%%%%%%%%%%%%%%%%

\begin{abstract}
\noindent 
This is a simple template for my papers. It's not very different from the plain article style, but it has most of the macros I use pre-written.
\end{abstract}



\small
\setcounter{tocdepth}{2}
\tableofcontents
\normalsize
%\clearpage


%%%%%%%%%%%%%%%%%%%%%
%%%  THE MEAT    %%%%
%%%%%%%%%%%%%%%%%%%%%

%% Use \input if you have separate files.
%% \include is `smarter' (creates separate aux files
%% for each tex file) and hence more efficient, 
%% but it automatically puts a page break
%% between included files. Input doesn't do this.


\section{Introduction}

In the beginning... \flip{a comment}


\section{Note on External TikZ Diagrams}

 This is a bit cumbersome, so I'll leave it commented out by default
 Usage: create the subdirectory tikz (or whatever ``prefix'' you use in
 the \texttt{tikzexternalize} option). Make sure the \texttt{-shell-escape} 
 is used when you compile. 
 %
 For Texpad this is a checkbox in the typesetting preferences.
 Tikz pictures will be exported as PDFs in the tikz directory.

 Note: this is NOT perfect! Some diagrams come out wonky, especially if you
 use arrows (e.g. for Feynman diagrams) or if you put the Tikz pictures in
 odd places (like in equation environments). What it does buy you is a HUGE
 improvement in compile time. I suggest using this for intermediate 
 typesets in a large document. For the final compile just turn it off again
 so you get clean graphics.
 
\section{Itemize test}

\begin{itemize}% \itemsep1pt \parskip0pt \parsep0pt
\item Item 1
\item Item 2
\item Item 3
\end{itemize}

\begin{enumerate}% \itemsep1pt \parskip0pt \parsep0pt
\item Item 1
\item Item 2
\item Item 3
\end{enumerate}

\section{Table Example}

 
\begin{table}
	\renewcommand{\arraystretch}{1.3} % spacing between rows
	\centering
	\begin{tabular}{ @{} llll @{} } \toprule % @{} removes space
		Element & Core MF & Mantle MF & $C_\text{cap}^N (\text{s}^{-1})$ 
		\\ \hline
		Iron & 0.855 & 0.0626 & $9.43\times 10^{7}$ 
		\\
		Nickel & 0.052 & 0.00196 & $7.10\times 10^{6}$ 
		\\
		Silicon & 0.06 & 0.210 & $2.24\times 10^{6}$ 
		\\
		Magnesium & 0 & 0.228 & $1.05\times 10^{6}$ 
		\\ \bottomrule
	\end{tabular}
	\caption{
		Mass fractions of the Earth's core and mantle.
		\label{table:elements}
	}
\end{table}


 

\section{XeLaTeX}

XeLaTeX lets you access local system fonts for use in \LaTeX. It's great for Beamer, but I don't recommend it in a regular paper. It doesn't play well with some useful macros like `blackboard math,' \url{http://tinyurl.com/a28hrle}.

\section{Other Remarks}

\begin{itemize}
	\item If you have problems with label references in captions, you may need to use the protect command. See \url{http://www.tex.ac.uk/FAQ-protect.html}
\end{itemize}

\section{lipsum}
 
 \lipsum[3-5]

\section*{Acknowledgements}


\acro{p.t.} is supported by the \acro{DOE} grant \textsc{de-sc}/0008541.
%
We thank
\emph{your name here}
for useful comments and discussions. 
%
\textsc{p.t.} thanks the Aspen Center for Physics (NSF grant \#1066293) for its hospitality during a period where part of this work was completed.

%% Appendices
% \appendix


%% Bibliography
\bibliographystyle{utcaps} 	% arXiv hyperlinks
% \bibliography{bib title without .bib}


\end{document}